\documentclass[report,a4paper,twocolumn]{memoir}
\usepackage[portuguese]{babel}
\usepackage[utf8]{inputenc}
\usepackage{hyperref}
\usepackage{ctable}
\usepackage{titlesec}
\usepackage[all]{hypcap}
\usepackage{geometry}
\usepackage{arevtext,arevmath}
% \usepackage{latexsym}
% \usepackage{enumitem}
% \usepackage{remreset}
% \usepackage{multirow}
% \usepackage{lscape}
% \usepackage{booktabs}

\usepackage{tikz}
% Define box and box title style
%\tikzstyle{mybox} = [draw, very thick, rectangle,rounded corners, inner sep=10pt, inner ysep=20pt]
% %%%%%%%%%%%%%%%%%%%%%%%%%%%%%%%%%%%%%%%%
% \usetikzlibrary{matrix}
% \usetikzlibrary{positioning}
% % % Fixing delimiter size for tikz matrices
% % \makeatletter
% % \def\tikz@delimiter#1#2#3#4#5#6#7#8{%
% %   \bgroup
% %   \pgfextra{\let\tikz@save@last@fig@name=\tikz@last@fig@name}%
% %   node[outer sep=0pt,inner sep=0pt,draw=none,fill=none,anchor=#1,at=(\tikz@last@fig@name.#2),#3]
% %   {%
% %     {\nullfont\pgf@process{\pgfpointdiff{\pgfpointanchor{\tikz@last@fig@name}{#4}}{\pgfpointanchor{\tikz@last@fig@name}{#5}}}}%
% %     \delimitershortfall\z@% as suggested by morbusg (maximum space not covered by a delimiter = 0)
% %     \resizebox*{!}{#8}{$\left#6\vcenter{\hrule height .5#8 depth .5#8 width0pt}\right#7$}%
% %   }
% %   \pgfextra{\global\let\tikz@last@fig@name=\tikz@save@last@fig@name}%
% %   \egroup%
% % }
% % \makeatother
% % \tikzset{
% %   withbrackets/.style = {draw, outer sep=0pt,
% %     left delimiter=[, right delimiter=],
% %     align=center}
% % }

% % %%%%%%%%%%%%%%%%%%%%%%%%%%%%%%%%%%%%%%%%

\renewcommand{\familydefault}{\sfdefault}

%\title{4\textordfeminine\ Conferência\\Latino Americana de\\Computação Científica e Python}

\titleformat{\chapter}[display]
{\bfseries\LARGE\sffamily}{\chaptertitlename}{10pt}{\LARGE}
\titlespacing*{\chapter}{0pt}{10pt}{10pt}
% \numberwithin{table}{chapter}

\begin{document}

\newgeometry{left=0cm, bottom=0cm, top=0cm, right=0cm}
\thispagestyle{empty}
\onecolumn
\begin{center}
  \includegraphics{imagens/capa.jpg}
\end{center}
\twocolumn

\clearpage

\restoregeometry
\pagestyle{plain} % No headers, just page numbers
\setcounter{page}{2}
\tableofcontents

\chapter*{Apresentação}

SciPy Latin America 2016 (SciPyLA 2016), a quarta conferência anual de Computação Científica com Python, será realizada nos dias 25, 26 e 27 de maio de 2016, em Florianópolis / Brasil.

A conferência SciPy Latin America é focada em aplicações científicas e afins que
utilizam Python, seguindo o exemplo de outras conferências regionais de mesmo tema que ocorrem na Europa (EuroSciPy), Índia (SciPy India) e Estados Unidos (SciPy). Esse encontro sucede 3 outros encontros de mesma temática ocorridos na Argentina em 2013, 2014 e 2015.

A SciPyLA 2016 consistirá de várias oficinas e palestras voltadas a pesquisadores, professores de diferentes níveis educacionais, estudantes, profissionais e empresários.

A comunidade SciPy é dedicada ao avanço da computação científica através de softwares \emph{Open Source} em Python para, mas não limitado a, Ciências Exatas, Biológicas, Humanas, e da Terra.

Tomando-se por base os encontros anteriores realizados na Argentina e Brasil, espera-se uma participação de aproximadamente 200 participantes provenientes das mais diferentes regiões. Na conferência palestrarão destacados pesquisadores de universidades Latinoamericanas e do mundo.

A organização da conferência e o auxílio de participação e assistência de pesquisadores e do comitê de organização geral, tanto brasileiros como de outros países,
%haverá
necessitará de
%
um considerável investimento que é normalmente financiado com o aporte de instituições públicas e privadas.

O êxito desta conferência é muito importante para a comunidade educacional, científica e industrial, e todos os demais influenciados por estas.

\begin{center}
\includegraphics[width=6cm]{imagens/IMG_20150521_102157-small.jpg}
\end{center}

\chapter*{Metodologia}

A conferência será voltada para a divulgação de ferramentas e/ou projetos realizados com Python no ambiente científico, acadêmico e industrial, e seus principais objetivos incluem:
\begin{itemize}
\item Dar oportunidade a divulgação da linguagem Python na comunidade científica Latinoamericana;
\item Divulgar e/ou revisitar ferramentas disponíveis aplicadas a problemas científicos ou outras de natureza similar;
\item Apresentar bibliotecas científicas desenvolvidas pela comunidade;
\item Combinar educação, engenharia e ciência através de Python;
\item Criar um marco apropriado para realização de encontros, fóruns de discussão, projetos educativos e de pesquisa relacionados a Python.
\end{itemize}

A organização e o desenvolvimento da conferência serão encaminhados por um comitê de docentes da Universidade Federal de Santa Catarina, entidades parceiras e membros da comunidade SciPy Latin América, localizados em diferentes países.

O comitê organizador local está constituído pelos seguintes membros:
\begin{itemize}
\item Melissa Weber Mendonça;
\item Antonio Kanaan;
\item Ivan Ogassavara;
\item Mário Sérgio Oliveira de Queiroz;
\item Filipe Pires Alvarenga Fernandes;
\item Horacio Andres Vargas Guzmán;
\item João Felipe Nicolaci Pimentel;
\item Matheus Braun Magrin;
\item Raniere Gaia Costa da Silva;
\end{itemize}

\begin{center}
\includegraphics[width=6cm]{imagens/IMG_20150521_102919-small.jpg}
\end{center}

\chapter*{Modalidades de apresentação}

Haverá
%4
cinco
%
modalidades de apresentação de trabalhos:
\begin{itemize}
\item Palestras;
\item Pôsteres;
\item Tutoriais;
\item Palestras relâmpago;
\item Track Teen.
\end{itemize}

\section*{Palestras}

São as tradicionais palestras oferecidas aos dias principais da conferência. Possuem duração máxima de 30 minutos com 5 minutos para perguntas. Se você acredita
que possui um tema mas não sabe como escrever a proposta, contate nosso comitê
de atividades e iremos lhe ajudar nessa tarefa. Adoraremos ajudá-lo a submeter
uma ótima proposta.

\section*{Tutoriais}

Estamos procurando por tutoriais que possam ajudar a comunidade a crescer em qualquer nível. Nossa meta é ter tutoriais que possam avançar a computação científica com Python, melhorar a comunidade e moldar o futuro. Tutoriais possuem duração de 100-120 minutos, mas se você acredita que precisa de mais de um \emph{slot}, você pode dividir o conteúdo e submeter duas propostas que se complementam, mas que são independentes.

\section*{Pôsteres}

A sessão de pôsteres oferece uma apresentação mais interativa e direcionada ao público do que as palestras. A ideia é apresentar seu tópico no pôster e quando os participantes trocam de salas eles encontram seu trabalho, leem o que você escreveu e iniciam uma discussão sobre ele. Simples assim. Você pode fazer uma sessão de perguntas e respostas nos primeiros minutos da sessão com um grupo de 10 pessoas.

\section*{Palestras relâmpago}

Deseja dar uma palestra, mas não tem material suficiente para uma? Essas palestras são de, no máximo, 5 minutos em uma sequência de palestras similares no salão principal. Não é preciso preencher todo os tempo disponível. A inscrição dessa modalidade será realizada durante o evento até a última apresentação do dia.

\section*{Track Teen}

O Track Teen é uma jornada que tem como propósito introduzir a um público de grande potencial criativo (como crianças e adolescentes) o mundo da ciência, programação e qualquer área que pode contribuir para o seu criativo, através de um ensino e diversão de maneira interativa.

Um dos objetivos desta conferência é mostrar às crianças que a ciência e a programação são realizadas por pessoas semelhantes a elas. Ambos estão presentes no uso diário de qualquer pessoa e a única coisa necessária é o anseio por aprender e implementar essas idéias que surgem em suas mentes. Sendo possível tornar-se geradores de idéias que podem ser úteis a outros.

\begin{center}
\includegraphics[width=6cm]{imagens/CFc8POiWYAEcPAV-small.jpg}
\end{center}

\chapter*{Público Alvo}

O público esperado é composto de estudantes e pesquisadores de universidades e empresas que utilizam ou tem interesse em utilizar Python em suas atividades. Espera-se um número máximo de 200 pessoas para o evento.

\begin{center}
\begin{tabular}{l c}
\bfseries{Público} & \bfseries{Esperado} \\\toprule
Professores/Pesquisadores & 50\\
Professores de Educação básica & 20\\
Alunos de Pós-Graduação & 30\\
Alunos de Graduação & 60\\
Profissionais & 40\\
Outros (discriminar) & 0\\\bottomrule
\end{tabular}
\end{center}

\begin{center}
\includegraphics[width=6cm]{imagens/CFlOZY5WYAETG6-small.jpg}
\end{center}

\chapter*{Cronograma Previsto}

As datas importantes do evento são listadas a seguir:
\begin{itemize}
\item 8 de Abril de 2016: Prazo final para submissão de palestras, pôsteres e
tutoriais.
\item 22 de Abril de 2016: Notificação de aceite para palestras, pôsteres e
tutoriais.
\item 23-24 de Maio de 2016: Pré-Evento
\item 25-27 de Maio de 2016: SciPy Latino-América 2016.
\item 28-29 de Maio de 2016: Pós-Evento
\end{itemize}

\chapter*{Coordenação Geral da Conferência SciPy Latin America 2016}

\begin{tabular}{l}
\bfseries{Coordenador}\\
Ivan Ogasawara\\
Telefone: (48) 9909\textendash 0207\\
\makeatletter email: ivan.ogasawara@gmail.com\makeatother \\
\\
\bfseries{Subcoordenador}\\
Raniere Gaia Costa da Silva\\
Telefone: (19) 9 8819\textendash 6817\\
\makeatletter email: raniere@rgaiacs.com \makeatother
\end{tabular}
\end{document}
